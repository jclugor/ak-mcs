
\chapter{Introduction}
\label{ch:1}

The study of the behavior of a structure given the properties of its materials, its geometric configuration, and the loads to which it is subjected is known as structural analysis. Within this field, it is of particular interest to determine when a structure exceeds certain levels of distress that can lead to collapse, considerably affect its performance, or affect its serviceability, among other criteria. These are known as limit states \citep{Melchers2018}. \\

The determination of whether these limits have been exceeded has been addressed from different perspectives throughout history. When the variables involved are considered to take unique values, we have what is known as the deterministic approach. There is a natural extension of the latter, which consists of considering that the values of these variables present variations derived from their intrinsic uncertainty, this being known as the probabilistic approach \citep{Ditlevsen1996}. \\

The term reliability is commonly defined as the complement of the probability of failure (failure being understood as the exceeding of a predefined limit state). In other words, reliability corresponds to the probability that the structure will function properly during a period of time \citep{Melchers2018}. \\

The use of reliability methods in structural engineering has several advantages, as stated in \citep{Lemaire2009}. Among them, there are:

\begin{itemize}
    \item offer a more realistic processing of uncertainties, allowing for a better understanding of these in the response of the structures, 
    \item allow an optimal distribution of materials and resources in each component as well as in the entire structure,
    \item permit the evaluation of the regulations established in building codes (e.g. safety factors), which can sometimes appear arbitrary due to the ignorance of the uncertainties associated with the phenomena
\end{itemize}

As a result, reliability analysis methods are being used more and more, given the stringent performance requirements, safety margins, and competitiveness of the market. They are particularly useful when dealing with unconventional structures for which there is insufficient empirical information. \citep{Choi2006}. \\

Several methods have been developed to estimate reliability, based on different principles. Each of them has advantages and disadvantages according to the nature of the problem, the number of variables involved and how they interact with each other, among other aspects. One of the most robust methods for reliability estimation is the Monte Carlo Simulation (MCS). The concept behind MCS is relatively simple, and it is a very powerful method, but its main drawback is that it can become computationally expensive \citep{Nowak2000}. \\

Various methods have been derived from MCS, trying to overcome the problem of
its computational cost, while preserving its strengths. In this document we study
one of those methods, known as AK-MCS for Active learning reliability method combining
Kriging and MCS, which attempts to obtain results equivalent to those given by MCS in a 
more efficient manner \citep{Echard2011}.

\section{Objectives}
The main objective of this final project is to implement the algorithm proposed by \citep{Echard2011} that employs MCS and a Kriging
metamodel to asses structural reliability. In order to satisfactorily do this, many
specific objectives have to be achieved, which include:
\begin{itemize}
    \item Learn the Monte Carlo Simulation method.
    \item Learn the Kriging regression algorithm.
    \item Completely understand the article.
    \item Implement the previously mentioned methods in Python.
    \item Review articles that aim to improve the performance of the AK-MCS algorithm.
    \item Evaluate the performance of the algorithm, comparing it with the basic MCS.
\end{itemize}

