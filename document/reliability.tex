\chapter{Basic concepts of Structural Reliability}
\label{ch:2}
Human understanding of the laws of nature allows him to try to model the physical phenomena
occurring around him. This knowledge will never be complete, but it can be attempted to be
taken as in-depth as desired. How rigorous one wants to be becomes a trade-off between
the benefits it brings and the difficulty and costs involved. \\

Regarding structures, it is of interest to study certain situations in which
undesirable conditions take place. These are called limit state violations.
The study of structural reliability concerns the calculation and prediction of
the probability of a structure reaching such states \citep{Melchers2018}. \\

In accordance with the previously mentioned, the estimation of structural reliability can be done from
different approaches, considering to a greater or lesser extent the uncertainties
inherent to structural problems. 

\section{Uncertainties in Reliability Assesment}
Uncertainties are what make the study of reliability meaningful. During the
service life of a structure, several variables are involved that introduce
uncertainties. A basic classification can be made according to their origin: \citep{Nowak2000}:
\begin{itemize}
    \item \textbf{Natural causes:} the unpredictability of loads\sidenote{such as
    earthquakes, water pressure, live load, etc} and the material properties
    that affects the mechanical behavior.
    \item \textbf{Human causes:} every task performed by a human being has a
    certain amount of uncertainty. Decision making is affected by many factors
    such as the level of knowledge, the presence of distractions, communication
    problems, among others.
\end{itemize}
Due to uncertainties, the loads and resistances of a structure can be considered
as random variables, and as such, they are a function of three factors:
\begin{itemize}
    \item \textbf{Physical variation:} the variation that is inherent in the 
    considered quantity.
    \item \textbf{Statistical variation:} parameter estimation based on a
    limited sample size leads to uncertainty.
    \item \textbf{Model variation:} simplifications made, unknown correlation 
    between variables, correlations and unknown boundary conditions.
\end{itemize}

\section{Deterministic approach}
The deterministic approach is the traditional one. In this approach, the uncertainties
in the forces and resistances are covered by assuming that the variation of these
forces and resistances have limits that are not exceeded. Given this simplicity,
it is usually necessary to overestimate the magnitudes in order to provide greater
safety. This is usual in design codes when considering safety factors in the limit
state equations. It is widely used, especially in simple and common problems.
However, this required safety margin implies the need to assume higher costs. 
\section{Partial probabilistic approach}
The historical record of events that expose structures to large loads, such as
earthquakes, floods, tornadoes, etc., shows that they have a certain frequency
according to their magnitude. For instance, during one day there may be multiple
low-scale earthquakes, but every few years there is one of considerable destructive
power. The partially probabilistic approach handles these return periods when considering
the maximum loads to which a structure will be subjected, since it is not cost-effective
to design an element for an event that it will probably never have to face. As its
name indicates, this approach considers some randomness of the variables involved,
but continues to make many simplifications.

\section{Probabilistic approach}
As discussed above, there is some randomness associated with the variables that
affect the loads and strengths of structures. Taking this into account, allows
for a deeper analysis, and therefore, to know in greater detail the behavior of
the structure, which is always desirable. \\

By making use of statistics, it is possible to infer an approximate distribution
that the values of the variables mentioned above follow. By considering this, it
is possible to estimate the probability that the loads are greater than the
resistances, being this the probability of failure of the structure. However,
in most cases this calculation is very complex and it is not possible to solve
it analytically.For this reason, it is necessary to use methods that provide an
approximation of the probability of failure.

\section{Monte Carlo Simulation}
With knowledge of the distributions followed by the variables involved in the
reliability problem, by means of numerical techniques it is possible to generate
random samples of numbers following these distributions. For a given sample it is
possible to estimate the average number of points in this space that cause the structure
to fail. By the law of large numbers, this average gets closer and closer to the expected
value of the distribution. The Monte Carlo simulation is based on this principle. Despite
its simplicity, this method is very robust when dealing with structural reliability problems.
