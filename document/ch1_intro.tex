
\chapter{Introduction}
\label{ch:1}

The study of the behavior of a structure given the properties of its materials, its geometric configuration, and the loads to which it is subjected is known as structural analysis. Within this field, it is of particular interest to determine when a structure exceeds certain levels of distress that can lead to collapse, considerably affect its performance, or affect its serviceability, among other criteria. These are known as limit states \citep{Melchers2018}. \\

The determination of whether these limits have been exceeded has been addressed from different perspectives throughout history. When the variables involved are considered to take unique values, we have what is known as the deterministic approach. There is a natural extension of the latter, which consists of considering that the values of these variables present variations derived from their intrinsic uncertainty, this being known as the probabilistic approach \citep{Ditlevsen1996}. \\

The term reliability is commonly defined as the complement of the probability of failure (failure being understood as the exceeding of a predefined limit state). In other words, reliability corresponds to the probability that the structure will function properly during a period of time \citep{Melchers2018}. \\

Several methods have been derived from MCS, trying to overcome the problem of
its computational cost, while preserving its strengths. In this document we study
one of those methods, known as AK-MCS for Active learning reliability method combining
Kriging and MCS, which attempts to obtain results equivalent to those given by MCS in a 
more efficient manner \citep{Echard2011}.

\section{Motivation}
The fundamental reason for doing this project is the desire to be initiated in academic
research, which contributes to the generation of knowledge and the development of 
science and the self.\\

Additionally, the structural reliability assesment is a very fascinating field which 
is constantly developing and can have a great impact. I can visualize myself working on these 
topics in the future.
\section{Objective}
The main objective of this final project is to implement the algorithm described in 
an article\cite{Echard2011} that proposes a new approach based on MCS and Kriging
metamodel to asses structural reliability. In order to satisfactorily do this, many
specific objectives have to be achieved, which include:
\begin{itemize}
    \item Learning the Monte Carlo Simulation method.
    \item Learning the Kriging regression algorithm.
    \item Reading and completely understand the article.
    \item Implementing the previously mentioned methods in Python.
    \item Reviewing articles that aim to improve the performance of the AK-MCS algorithm.
    \item Evaluating the performance of the algorithm, comparing it with the basic MCS.
\end{itemize}

