
\chapter{Introduction}
\label{ch:1}

The human being's desire of self-improvement leads him to try to do things
better over time. Among its many creations, buildings and tools have played a
fundamental role in the development of mankind. However, this development is
constrained by the availability of resources (e.g. natural, economic), so their efficient
use is becoming increasingly important. It is at this point where a deeper knowledge
and interpretation of physical phenomena can significantly improve the design and
construction of the aforementioned structures. \\

Deterministic models usually require a greater use of resources to deal with the uncertainty
present in structural systems. This may not have as much impact when dealing with 
simple structures, but when it comes to the more and more common highly complex ones, the cost
overruns can be quite considerable. In order to avoid this situation, a probabilistic
approach is a more suitable option \citep{Choi2006}.\\

When using the probabilistic approach, reliability is understood as the probability
that a structure will not fail, referring as failure to the structure not performing
as desired. One of the most important technique to solve structural reliability problems
is the Monte Carlo Simulation (MCS). The concept behind MCS is relatively simple, and
it is a very powerful method, but its main drawback is that it can become computationally
expensive \citep{Nowak2000}. \\

Several methods have been derived from MCS, trying to overcome the problem of
its computational cost, while preserving its strengths. In this document we study
one of those methods, known as AK-MCS for Active learning reliability method combining
Kriging and MCS, which attempts to obtain results equivalent to those given by MCS in a 
more efficient manner \citep{Echard2011}.

\section{Motivation}
The fundamental reason for doing this project is the desire to be initiated in academic
research, which contributes to the generation of knowledge and the development of 
science and the self.\\

Additionally, the structural reliability assesment is a very fascinating field which 
is constantly developing and can have a great impact. I can visualize myself working on these 
topics in the future.
\section{Objective}
The main objective of this final project is to implement the algorithm described in 
an article\cite{Echard2011} that proposes a new approach based on MCS and Kriging
metamodel to asses structural reliability. In order to satisfactorily do this, many
specific objectives have to be achieved, which include:
\begin{itemize}
    \item Learning the Monte Carlo Simulation method.
    \item Learning the Kriging regression algorithm.
    \item Reading and completely understand the article.
    \item Implementing the previously mentioned methods in Python.
    \item Reviewing articles that aim to improve the performance of the AK-MCS algorithm.
    \item Evaluating the performance of the algorithm, comparing it with the basic MCS.
\end{itemize}

